\ifx\wholebook\relax \else
% ------------------------

\documentclass[UTF8]{article}
%------------------- Other types of document example ------------------------
%
%\documentclass[twocolumn]{IEEEtran-new}
%\documentclass[12pt,twoside,draft]{IEEEtran}
%\documentstyle[9pt,twocolumn,technote,twoside]{IEEEtran}
%
%-----------------------------------------------------------------------------
%
% loading packages
%

\RequirePackage{ifpdf}
\RequirePackage{ifxetex}

%
%
\ifpdf
  \RequirePackage[pdftex,%
       bookmarksnumbered,%
              colorlinks,%
          linkcolor=blue,%
              hyperindex,%
        plainpages=false,%
       pdfstartview=FitH]{hyperref}
\else\ifxetex
  \RequirePackage[bookmarksnumbered,%
               colorlinks,%
           linkcolor=blue,%
               hyperindex,%
         plainpages=false,%
        pdfstartview=FitH]{hyperref}
\else
  \RequirePackage[dvipdfm,%
        bookmarksnumbered,%
               colorlinks,%
           linkcolor=blue,%
               hyperindex,%
         plainpages=false,%
        pdfstartview=FitH]{hyperref}
\fi\fi
%\usepackage{hyperref}

% other packages
%--------------------------------------------------------------------------
\usepackage{graphicx, color}
\usepackage{subfig}
\usepackage{multicol}
\usepackage{tikz}
\usetikzlibrary{matrix,positioning,shapes}

\usepackage{amsmath, amsthm, amssymb} % for math
\usepackage{exercise} % for exercise
\usepackage{import} % for nested input

%
% for programming
%
\usepackage{verbatim}
\usepackage{fancyvrb}
\usepackage{listings}
%\usepackage{algorithmic} %old version; we can use algorithmicx instead
%\usepackage[plain]{algorithm} %remove rule (horizontal line on top/below the algorithm
\usepackage{algorithm} %to remove rules change to \usepackage[plain]{algorithm}
%\usepackage{algorithm2e}
\usepackage[noend]{algpseudocode} %for pseudo code, include algorithmicsx automatically
\usepackage{appendix}
\usepackage{makeidx} % for index support
\usepackage{titlesec}

\usepackage[cm-default]{fontspec}
\usepackage{xunicode}
%\usepackage{fontenc}
\usepackage{textcomp}
\usepackage{url}

% detect and select Chinese font
% ------------------------------
% the following cmd can list all availabe Chinese fonts in host.
% fc-list :lang=zh
\def\myfont{STSong}  % Under Mac OS X
\def\linuxfallback{WenQuanYi Micro Hei} % Under Linux
\def\winfallback{SimSun} % Under Windows
\suppressfontnotfounderror1 % Avoid setting exit code (error level) to break make process
\count255=\interactionmode
\batchmode
\font\foo="\myfont"\space at 10pt
\ifx\foo\nullfont
  \font\foo = "\linuxfallback"\space at 10pt
  \ifx\foo\nullfont
    \font\foo = "\winfallback"\space at 10pt
    \ifx\foo\nullfont
      \errorstopmode
      \errmessage{no suitable Chinese font found}
    \else
      \let\myfont=\winfallback % Windows
    \fi
  \else
    \let\myfont=\linuxfallback % Linux
  \fi
\fi
\interactionmode=\count255
\setmainfont[Mapping=tex-text]{\myfont}
\setmonofont[Scale=MatchLowercase]{Monaco}   % 英文等宽字体

\XeTeXlinebreaklocale "zh"  % to solve the line breaking issue
\XeTeXlinebreakskip = 0pt plus 1pt minus 0.1pt

\titleformat{\paragraph}
{\normalfont\normalsize\bfseries}{\theparagraph}{1em}{}
\titlespacing*{\paragraph}
{0pt}{3.25ex plus 1ex minus .2ex}{1.5ex plus .2ex}

\lstdefinelanguage{Smalltalk}{
  morekeywords={self,super,true,false,nil,thisContext}, % This is overkill
  morestring=[d]',
  morecomment=[s]{"}{"},
  alsoletter={\#:},
  escapechar={!},
  literate=
    {BANG}{!}1
    {UNDERSCORE}{\_}1
    {\\st}{Smalltalk}9 % convenience -- in case \st occurs in code
    % {'}{{\textquotesingle}}1 % replaced by upquote=true in \lstset
    {_}{{$\leftarrow$}}1
    {>>>}{{\sep}}1
    {^}{{$\uparrow$}}1
    {~}{{$\sim$}}1
    {-}{{\sf -\hspace{-0.13em}-}}1  % the goal is to make - the same width as +
    %{+}{\raisebox{0.08ex}{+}}1		% and to raise + off the baseline to match -
    {-->}{{\quad$\longrightarrow$\quad}}3
	, % Don't forget the comma at the end!
  tabsize=2
}[keywords,comments,strings]

% for literate Haskell code
\lstdefinestyle{Haskell}{
  flexiblecolumns=false,
  basewidth={0.5em,0.45em},
  morecomment=[l]--,
  literate={+}{{$+$}}1 {/}{{$/$}}1 {*}{{$*$}}1 {=}{{$=$}}1
           {>}{{$>$}}1 {<}{{$<$}}1 {\\}{{$\lambda$}}1
           {\\\\}{{\char`\\\char`\\}}1
           {->}{{$\rightarrow$}}2 {>=}{{$\geq$}}2 {<-}{{$\leftarrow$}}2
           {<=}{{$\leq$}}2 {=>}{{$\Rightarrow$}}2
           {\ .}{{$\circ$}}2 {\ .\ }{{$\circ$}}2
           {>>}{{>>}}2 {>>=}{{>>=}}2
           {|}{{$\mid$}}1
}

\lstloadlanguages{C, C++, Lisp, Haskell, Python, Smalltalk}

\lstset{
  basicstyle=\small\ttfamily,
  commentstyle=\rmfamily,
  texcl=true,
  showstringspaces = false,
  upquote=true,
  flexiblecolumns=false
}

% ======================================================================

\def\BibTeX{{\rm B\kern-.05em{\sc i\kern-.025em b}\kern-.08em
    T\kern-.1667em\lower.7ex\hbox{E}\kern-.125emX}}

%
% mathematics
%
\newcommand{\be}{\begin{equation}}
\newcommand{\ee}{\end{equation}}
\newcommand{\bmat}[1]{\left( \begin{array}{#1} }
\newcommand{\emat}{\end{array} \right) }
\newcommand{\VEC}[1]{\mbox{\boldmath $#1$}}

% numbered equation array
\newcommand{\bea}{\begin{eqnarray}}
\newcommand{\eea}{\end{eqnarray}}

% equation array not numbered
\newcommand{\bean}{\begin{eqnarray*}}
\newcommand{\eean}{\end{eqnarray*}}

\newtheorem{theorem}{定理}[section]
\newtheorem{lemma}[theorem]{引理}
\newtheorem{proposition}[theorem]{Proposition}
\newtheorem{corollary}[theorem]{Corollary}

% 中文书籍设置
% ====================================
\renewcommand\contentsname{目\ 录}
%\renewcommand\listfigurename{插图目录}
%\renewcommand\listtablename{表格目录}
\renewcommand\figurename{图}
\renewcommand\tablename{表}
\renewcommand\proofname{证明}
\renewcommand\ExerciseName{练习}
%\renewcommand{\algorithmcfname}{算法}

\ifx\wholebook\relax
\renewcommand\bibname{参\ 考\ 文\ 献}                    %book类型
%\newtheorem{Definition}[Theorem]{定义}
\newtheorem{Theorem}{定理}[chapter]
\newtheorem{example}{例题}[chapter]
\else
\renewcommand\refname{参\ 考\ 文\ 献}
\fi

%\setcounter{secnumdepth}{4}
\titleformat{\chapter}
  {\normalfont\bfseries\Large}
  {第\arabic{chapter}章}
  {12pt}{\Large}
%% \titleformat{\subsection}
%%   {\normalfont\bfseries\large}
%%   {\CJKnumber{\arabic{subsection}}、}
%%   {12pt}{\large}
%% \titleformat{\subsubsection}
%%   {\normalfont\bfseries\normalsize}
%%   {\arabic{subsubsection}.}
%%   {12pt}{\normalsize}

%\renewcommand{\baselinestretch}{1.5}                        %文章行间距为1.5倍。

\makeatletter
\newcommand{\verbatimfont}[1]{\renewcommand{\verbatim@font}{\ttfamily#1}}
\makeatother

\setcounter{tocdepth}{4}
\setcounter{secnumdepth}{4}

%\verbatimfont{\footnotesize}


\setcounter{page}{1}

\begin{document}

%--------------------------

% ================================================================
%                 COVER PAGE
% ================================================================

\title{AVL树}

\author{刘新宇
\thanks{{\bfseries 刘新宇} \newline
  Email: liuxinyu95@gmail.com \newline}
  }

\maketitle
\fi

\markboth{AVL树}{初等算法}

\ifx\wholebook\relax
\chapter{AVL树}
\numberwithin{Exercise}{chapter}
\fi

% ================================================================
%                 Introduction
% ================================================================

\label{introduction} \index{AVL树}

本章介绍AVL树。同红黑树类似,AVL树也是为了解决二叉树的平衡问题而提出的。它采用了更为直观的平衡性定义,以及恢复平衡的策略。对比红黑树和AVL树的设计和实现,有助于我们进一步理解自平衡二叉树的特性。

%\subsection{度量树的平衡性}

除了红黑树,还有没有其他自平衡二叉树呢?为了度量一棵二叉树的平衡,我们可以比较左右分支的高度差,如果差很大,则说明树不平衡。定义一棵树的高度差如下:

\be
  \delta(T) = |R| - |L|
\ee

其中$|T|$代表树$T$的高度,$L$和$R$分别代表左右分支。

若$\delta(T) = 0$,说明树是平衡的。例如,一棵高度为$h$的完全二叉树有$n = 2^h-1$个节点。除了叶子节点外,所有节点都含有两个非空的分支。完全二叉树的所有分支都满足$\delta(T)=0$。另外一个特殊的例子是空树:$\delta(\phi) = 0$。通常$\delta(T)$的绝对值越小,说明树越平衡。

我们定义$\delta(T)$为一棵二叉树的\underline{平衡因子}。

% ================================================================
% Definition
% ================================================================
\section{AVL树的定义}
\index{AVL树!定义}

如果一棵二叉搜索树的所有子树都满足如下条件,我们称之为AVL树。

\be
  |\delta(T)| \leq 1
\ee

AVL树中所有子树平衡因子的绝对值都不大于1,只可能是-1、0,或1这三个值。图\ref{fig:avl-example}给出了一棵AVL树的例子。

\begin{figure}[htbp]
   \centering
   \includegraphics[scale=0.5]{img/avl-example.ps}
   \caption{AVL树的例子} \label{fig:avl-example}
\end{figure}

为什么AVL树能保证平衡性呢?或者说为什么这个定义能保证一棵有$n$个节点的树的高度为$O(\lg n)$?我们可以用下面的方法来证明这一事实。

对于一棵高为$h$的AVL树,它的节点数目并不是一个固定的值。当它是一棵完全二叉树时,含有的节点数目最多,为$2^h-1$。那么它最少包含多少节点呢?定义函数$N(h)$代表高度为$h$的AVL树所含有的最少节点数目。对于简单的情况,我们可以立即得出$N(h)$的值:

\begin{itemize}
\item 空树,$h=0$,$N(0)=0$;
\item 只有一个根节点的树,$h=1$,$N(1)=1$;
\end{itemize}

一般情况下$N(h)$是怎样的?图\ref{fig:N-h-relation}中给出了一个高度为$h$的AVL树$T$。它包含三部份:根节点和左右两个分支$L$与$R$。树的高度和子树高度之间满足下面的关系:

\be
  h= max(|L|, |R|) + 1
\ee

因此,必然存在一个子树的高度为$h-1$。根据AVL树的定义,我们有 $||L| -|R|| \leq 1$。所以另外一棵子树的高度不会小于$h-2$。而$T$所包含的节点数为两个子树的节点数再加1(1个根节点)。于是我们得到下面的递归关系:

\be
  N(h) = N(h-1) + N(h-2) + 1
  \label{eq:Fibonacci-like}
\ee

\begin{figure}[htbp]
   \centering
   \includegraphics[scale=0.5]{img/Nh-lvr.ps}
   \caption{高度为$h$的AVL树,其中一个分支高$h-1$,另外一个分支的高度不小于$h-2$。} \label{fig:N-h-relation}
\end{figure}

这一递归形式让我们联想起著名的斐波那契(Fibonacci)数列。如果定义$N'(h) = N(h)+1$,我们就可以将(\ref{eq:Fibonacci-like})转换成斐波那契数列。

\be
  N'(h) = N'(h-1) + N'(h-2)
\ee

\begin{lemma}
\label{lemma:N-phi}
若$N(h)$表示高为$h$的AVL树的节点数目最小值,令$N'(h) = N(h) + 1$,则:

\be
  N'(h) \geq \phi^h
\ee

其中$\phi = \frac{\sqrt{5}+1}{2}$,通常被称为黄金分割比。
\end{lemma}

\begin{proof}[证明]
使用数学归纳法。对于起始情况,我们有:
\begin{itemize}
\item $h=0$, $N'(0) = 1 \geq \phi^0 = 1$
\item $h=1$, $N'(1) = 2 \geq \phi^1 = 1.618...$
\end{itemize}

对于递推情况,设$N'(h) \geq \phi^h$。
\[
  \begin{array}{lll}
  N'(h+1) & = N'(h) + N'(h-1) & \{Fibonacci\} \\
          & \geq \phi^h + \phi^{h-1} & \\
          & = \phi^{h-1}(\phi + 1) & \{\phi + 1 = \phi^2 = \frac{\sqrt{5}+3}{2}\} \\
          & = \phi^{h+1}
 \end{array}
\]
\end{proof}

由定理\ref{lemma:N-phi},我们立即得到下面的结果:

\be
  h \leq log_{\phi}(n+1) = log_{\phi}2 \cdot \lg (n+1) \approx 1.44 \lg (n+1)
  \label{eq:AVL-height}
\ee

这一不等式说明AVL树的高度为$O(\lg n)$,从而保证了平衡性。

在树的基本操作中,插入和删除会改变树的结构。如果由此导致平衡因子的绝对值超过1,就需要通过修复使得$|\delta|$恢复到1以内。常见的修复方法是使用树旋转。受到Okasaki在红黑树\cite{okasaki}中的思路启发,本章中,我们介绍一种模式匹配(pattern matching)方法。这种“改变―恢复”的操作,使得AVL树成为了一种自平衡二叉树。作为比较,本章同样也给出命令式的AVL树算法。

平衡因子$\delta$显然可以通过递归求出。另外一种方法是在每个节点中保存一分平衡因子的值,如果树结构发生改变,我们只要更新这个值就可以了。这一方法不需要每次都进行遍历计算。

根据这一思路,我们在二叉搜索树的定义中增加一个$\delta$变量,如下面的C++代码所示\footnote{有些实现不保存$\delta$,取而代之保存树的高度,如\cite{py-avl}。}。

\lstset{language=C++}
\begin{lstlisting}
template <class T>
struct node{
  int delta;
  T key;
  node* left;
  node* right;
  node* parent;
};
\end{lstlisting}

某些纯函数式实现使用不同的构造函数(constructor)来保存平衡因子$\delta$。例如在\cite{hackage-avl}中,定义了4个constructor:\texttt{E}、\texttt{N}、\texttt{P}、\texttt{Z}。其中,\texttt{E}代表空树$\phi$;\texttt{N}代表平衡因子为-1;\texttt{P}代表平衡因子为1;\texttt{Z}代表平衡因子为0。

本章中,我们直接在节点中保存平衡因子的值。

\lstset{language=Haskell}
\begin{lstlisting}
data AVLTree a = Empty
               | Br (AVLTree a) a (AVLTree a) Int
\end{lstlisting}

我们将略过树的只读操作,包括查找、寻找最大、最小值等等,它们和二叉搜索树完全一样。我们仅关注哪些会改变树结构的操作。

% ================================================================
%                 Insertion
% ================================================================
\section{插入}
\index{AVL树!插入}

在AVL树中插入一个新元素可能会破坏平衡,使得平衡因子$\delta$的绝对值超过1。为了恢复平衡,可以根据不同的情形进行树的旋转操作。大多数命令式实现采用这种方法。

另外一种方法很像Okasaki在红黑树实现中使用的模式匹配方法。它的特点是简单直观。

向AVL树中插入一个新key,根节点的平衡因子的\underline{变化}会在$[-1, 1]$之间\footnote{注意:这里不是说平衡因子的值在$[-1, 1]$内,而是说它的变化在这个范围内。},树的高度最多增加1。我们需要递归地使用这一信息来更新其他层级上的平衡因子。定义插入算法的结果为一对值$(T', \Delta H)$,其中$T'$为插入后的新树,$\Delta H$为树高度的增加值。令函数$first(pair)$取得一对值中的第一个元素,我们可以在二叉搜索树的插入算法上进行改动,定义AVL树的插入操作:

\be
insert(T, k) = first(ins(T, k))
\ee

其中

\be
ins(T, k) = \left \{
  \begin{array}
  {r@{\quad:\quad}l}
  ((\phi, k, \phi, 0), 1) & T = \phi \\
  tree(ins(L, k), k', (R, 0), \Delta) & k < k' \\
  tree((L, 0), k', ins(R, k), \Delta) & otherwise
  \end{array}
\right.
\label{eq:ins}
\ee

$L$、$R$、$k'$、$\Delta$的定义如下,它们分别表示左右子树,key和平衡因子。

\[
  \begin{array}{l}
  L = left(T) \\
  R = right(T) \\
  k' = key(T) \\
  \Delta = \delta(T)
  \end{array}
\]

向AVL树$T$中插入一个新key $k$时,如果树为空,结果为一个叶子节点,节点的key为$k$,平衡因子为0。树的高度增加1。

否则,如果$T$不为空,我们需要比较根节点的key $k'$和待插入key $k$的大小。如果$k$小于根节点的key,我们将其递归插入左子树,否则将其插入右子树。

根据定义,递归插入的结果为一对值,例如$(L', \Delta H_l)$。我们需要对插入的结果调整平衡,并更新高度的增加值。为此定义函数$tree()$,它接受4个参数:$(L', \Delta H_l)$、$k'$、$(R', \Delta H_r)$和$\Delta$。这一函数的运算结果记为$(T', \Delta H)$。其中,$T'$为调整平衡后的树,$\Delta H$是树高度的增加值,定义如下:

\be
  \Delta H = |T'| - |T|
\ee

它可以进一步分解为4种情况。

\be
\begin{array}{rl}
  \Delta H & = |T'| - |T| \\
              & = 1 + max(|R'|, |L'|) - (1 + max(|R|, |L|)) \\
              & = max(|R'|, |L'|) - max(|R|, |L|) \\
              & = \left \{
                  \begin{array}{r@{\quad:\quad}l}
                  \Delta H_r & \Delta \geq 0 \land \Delta' \geq 0 \\
                  \Delta + \Delta H_r & \Delta \leq 0 \land \Delta' \geq 0 \\
                  \Delta H_l - \Delta & \Delta \geq 0 \land \Delta' \leq 0 \\
                  \Delta H_l & otherwise
                  \end{array} \right .
\end{array}
\ee

由于一次插入操作不可能同时增加左右分支的高度,因此我们可以做上述分解。根据定义,平衡因子等于右子树的高度减去左子树的高度。这4种情况可以分别解释如下:

\begin{itemize}
\item 如果$\Delta \geq 0$并且$\Delta' \geq 0$。这说明在插入前后,右子树的高度都不小于左子树的高度。因子整个树高度的增加,全部“贡献”自右子树高度的变化$\Delta H_r$;

\item 如果$\Delta \leq 0$,说明在插入前,左子树的高度不小于右子树。但是插入后$\Delta' \geq 0$,说明右子树的高度由于插入操作增加了,而左子树的高度保持不变($|L'|=|L|$)。所以高度的增加为:
\[
\begin{array}{rll}
\Delta H & = max(|R'|, |L'|) - max (|R|, |L|) & \{\Delta \leq 0 \land \Delta' \geq 0 \}\\
         & = |R'|-|L| & \{|L|=|L'| \}\\
         & = |R|+\Delta H_r - |L| & \\
         & = \Delta + \Delta H_r &
\end{array}
\]

\item 如果$\Delta \geq 0$且$\Delta' \leq 0$,和第二种情况类似,我们有:

\[
\begin{array}{rll}
\Delta H & = max(|R'|, |L'|) - max (|R|, |L|) & \{\Delta \geq 0 \land \Delta' \leq 0 \}\\
         & = |L'|-|R| & \\
         & = |L|+\Delta H_l - |R| & \\
         & = \Delta H_l - \Delta&
\end{array}
\]

\item 最后一种情况,$\Delta$和$\Delta'$都不大于0,说明插入前后左子树的高度都不小于右子树。所以高度的增加全部“贡献”自左子树的变化$\Delta H_l$。
\end{itemize}

在进行平衡调整前,我们还需要确定新的平衡因子$\Delta'$。根据AVL树平衡因子的定义,我们有:

\be
\begin{array}{rl}
\Delta' & = |R'| - |L'| \\
        & = |R| + \Delta H_r - (|L| + \Delta H_l) \\
        & = |R| - |L| + \Delta H_r - \Delta H_l \\
        & = \Delta + \Delta H_r - \Delta H_l
\end{array}
\ee

树高度的变化和平衡因子都准备好后,就可以定义(\ref{eq:ins})中的函数$tree()$了。

\be
tree((L', \Delta H_l), k', (R', \Delta H_r), \Delta) =
  balance ((L', k', R', \Delta'), \Delta H)
\ee

在具体解释平衡调整的细节前,我们可以先给出上述函数的Haskell例子代码。首先是插入函数:

\lstset{language=Haskell}
\begin{lstlisting}
insert t x = fst $ ins t where
    ins Empty = (Br Empty x Empty 0, 1)
    ins (Br l k r d)
        | x < k     = tree (ins l) k (r, 0) d
        | x == k    = (Br l k r d, 0)
        | otherwise = tree (l, 0) k (ins r) d
\end{lstlisting} %$

这段代码中,如果待插入的key已经存在,它仅仅使用新key覆盖原先的值。

\begin{lstlisting}
tree (l, dl) k (r, dr) d = balance (Br l k r d', delta) where
    d' = d + dr - dl
    delta = deltaH d d' dl dr
\end{lstlisting}

高度增加的计算函数定义如下:

\begin{lstlisting}
deltaH d d' dl dr
       | d >=0 && d' >=0 = dr
       | d <=0 && d' >=0 = d+dr
       | d >=0 && d' <=0 = dl - d
       | otherwise = dl
\end{lstlisting}

\subsection{平衡调整}
\index{AVL树!平衡调整}
我们准备使用模式匹配(pattern matching)来恢复平衡,首先需要考虑有哪些情况(pattern)会破坏AVL树的性质。

图\ref{fig:avl-insert-fix}中展示了4种需要修复平衡的情况。这些情况中,平衡因子都是2或者-2,而不在范围$[-1, 1]$之内。通过调整,平衡因子变成0,左右分支变成同样的高度。

\begin{figure}[htbp]
   \begin{center}
     \setlength{\unitlength}{1cm}
     \begin{picture}(15, 15)
        % graphics
        \put(0, 7){\includegraphics[scale=0.5]{img/avl-insert-ll.ps}}
        \put(0, 0){\includegraphics[scale=0.5]{img/avl-insert-lr.ps}}
        \put(7, 7){\includegraphics[scale=0.5]{img/avl-insert-rr.ps}}
        \put(8.5, 0){\includegraphics[scale=0.5]{img/avl-insert-rl.ps}}
        \put(2, 5){\includegraphics[scale=0.5]{img/avl-insert-fixed.ps}}
        % arrows
        \put(4.5, 9.5){\vector(1, -1){1}}
        \put(4.5, 5){\vector(1, 1){1}}
        \put(10, 9.5){\vector(-1, -1){1}}
        \put(10, 5){\vector(-1, 1){1}}
        % delta values
        \put(5, 13){$\delta(z) = -2$}
        \put(2.5, 12){$\delta(y) = -1$}
        \put(10, 13){$\delta(x) = 2$}
        \put(11.5, 11.5){$\delta(y) = 1$}
        \put(1.5, 5.5){$\delta(z) = -2$}
        \put(3.8, 3.8){$\delta(x) = 1$}
        \put(12, 5.5){$\delta(x) = 2$}
        \put(10, 3.5){$\delta(z) = -1$}
        \put(7, 9.5){$\delta'(y) = 0$}
      \end{picture}
     \caption{插入后需要调整平衡的4种情况。} \label{fig:avl-insert-fix}
  \end{center}
\end{figure}

我们从左上角开始,按照顺时针方向,依次称这4种情况为左-左偏(left-left lean)、右-右偏(right-right lean)、右-左偏(right-left lean)和左-右偏(left-right lean)。记调整前的平衡因子为$\delta(x)$、$\delta(y)$和$\delta(z)$;调整后的平衡因子为$\delta'(x)$、$\delta'(y)$和$\delta'(z)$。

我们接下来将证明,经过调整后,所有4种情况的平衡因子都变成$\delta(y)=0$。并且将给出调整后$\delta'(x)$和$\delta'(z)$的结果。

\subsubsection{左-左偏(Left-left lean)的情况}

由于$x$子分支在调整前后的结构维持不变,因此可以立即得到等式:$\delta'(x) = \delta(x)$。

因为$\delta(y) = -1$且$\delta(z) = -2$,所以:

\be
  \begin{array}{l}
  \delta(y) = |C| - |x| = -1 \Rightarrow |C| = |x| - 1 \\
  \delta(z) = |D| - |y| = -2 \Rightarrow |D| = |y| - 2
  \end{array}
  \label{eq:ll-cd}
\ee

调整平衡后:

\be
  \begin{array}{rll}
  \delta'(z) & = |D| - |C| & \{ \text{根据式} (\ref{eq:ll-cd}) \}\\
             & = |y| - 2 - (|x| - 1) & \\
             & = |y| - |x| - 1 & \{  x \text{是} y \text{的子节点} \Rightarrow |y|-|x| = 1\} \\
             & = 0 &
  \end{array}
  \label{eq:ll-delta-z}
\ee

对于$\delta'(y)$,调整平衡后我们有如下结果:

\be
  \begin{array}{rll}
  \delta'(y) & = |z| - |x| & \\
             & = 1 + max(|C|, |D|) - |x| & \{ \text{根据式(\ref{eq:ll-delta-z}),我们有} |C| = |D|\} \\
             & = 1 + |C| - |x| & \{ \text{根据式(\ref{eq:ll-cd})}\} \\
             & = 1 + |x| - 1 - |x| & \\
             & = 0 &
  \end{array}
\ee

汇总上述结果,对于左-左偏的情况,新的平衡因子如下:

\be
  \begin{array}{l}
  \delta'(x) = \delta(x) \\
  \delta'(y) = 0 \\
  \delta'(z) = 0
  \end{array}
\ee

\subsubsection{右-右偏(Right-right lean)的情况}

因为右-右偏和左-左偏对称,易知新的平衡因子结果如下:

\be
  \begin{array}{l}
  \delta'(x) = 0 \\
  \delta'(y) = 0 \\
  \delta'(z) = \delta(z)
  \end{array}
  \label{eq:rr-result}
\ee

\subsubsection{右-左偏(Right-left lean)的情况}

首先考虑$\delta'(x)$。调整平衡后,我们有:

\be
  \delta'(x) = |B| - |A|
  \label{eq:rl-dx}
\ee

调整平衡前,如果我们计算$z$的高度,有如下的结果:

\be
  \begin{array}{rll}
  |z| & = 1 + max(|y|, |D|) &  \{ \delta(z) = -1 \Rightarrow |y| > |D|\} \\
      & = 1 + |y| & \\
      & = 2 + max(|B|, |C|)
  \end{array}
  \label{eq:rl-z}
\ee

因为$\delta(x) = 2$,所以可以推出:

\be
  \begin{array}{rll}
  \delta(x) = 2 & \Rightarrow |z| - |A| = 2 & \{ \text{根据式(\ref{eq:rl-z})} \}\\
                & \Rightarrow 2 + max(|B|, |C|) - |A| = 2 & \\
                & \Rightarrow max(|B|, |C|) - |A| = 0 &
  \end{array}
  \label{eq:rl-ca}
\ee

如果$\delta(y) = 1$,也就是$|C| - |B| = 1$,则有下面的关系:

\be
  max(|B|, |C|)= |C| = |B|+1
\ee

将其代入式(\ref{eq:rl-ca})得到:

\be
  \begin{array}{ll}
  |B|+1-|A| = 0 \Rightarrow |B|-|A|= -1 & \{ \text{根据式(\ref{eq:rl-dx}) } \} \\
  \Rightarrow \delta'(x) = -1 &
  \end{array}
\ee

反之,如果$\delta(y) \neq 1$,则有$max(|B|, |C|) = |B|$,将其代入式(\ref{eq:rl-ca})得到:

\be
  \begin{array}{ll}
  |B| - |A| = 0  & \{ \text{根据式(\ref{eq:rl-dx})} \} \\
  \Rightarrow \delta'(x) = 0 &
  \end{array}
\ee

合并上述两种子情况,我们可以得到$\delta'(x)$和$\delta(y)$的关系:

\be
\delta'(x) = \left \{
  \begin{array}
  {r@{\quad:\quad}l}
  -1 & \delta(y) = 1 \\
  0 & otherwise
  \end{array}
\right.
\label{eq:rl-dx-dy}
\ee

对于$\delta'(z)$,根据定义,它等于:

\be
  \begin{array}{rll}
    \delta'(z) & = |D| - |C| & \{ \delta(z) = -1 = |D| - |y| \} \\
               & = |y| - |C| - 1 & \{ |y| = 1 + max(|B|, |C|) \} \\
               & = max(|B|, |C|) - |C|
  \end{array}
  \label{eq:rl-dz}
\ee

如果$\delta(y) = -1$,则有$|C| - |B| = -1$,所以$max(|B|, |C|) = |B| = |C| + 1$。将其代入式(\ref{eq:rl-dz})中,我们有:$\delta'(z) = 1$。

反之,如果$\delta(y) \neq -1$,则$max(|B|, |C|) = |C|$,我们有$\delta'(z)=0$。

合并上述两种子情况,$\delta'(z)$和$\delta(y)$的关系如下:

\be
\delta'(z) = \left \{
  \begin{array}
  {r@{\quad:\quad}l}
  1 & \delta(y) = -1 \\
  0 & otherwise
  \end{array}
  \right.
  \label{eq:rl-dz-dy}
\ee

最后,对于$\delta'(y)$,我们可以推导出下面的关系:

\be
  \begin{array}{rl}
  \delta'(y) & = |z| - |x| \\
             & = max(|C|, |D|) - max(|A|, |B|)
  \end{array}
  \label{eq:rl-dy}
\ee

这里又分为3种子情况:
\begin{itemize}

\item 若$\delta(y)=0$,说明$|B|=|C|$,根据式(\ref{eq:rl-dx-dy})和式(\ref{eq:rl-dz-dy}),我们有:$\delta'(x)=0 \Rightarrow |A| = |B|$以及$\delta'(z)=0 \Rightarrow |C|=|D|$。因此$\delta'(y)=0$。

\item 若$\delta(y)=1$,根据式(\ref{eq:rl-dz-dy}),我们有$\delta'(z)=0 \Rightarrow |C| = |D|$。

\[
  \begin{array}{rll}
  \delta'(y) & = max(|C|, |D|) - max(|A|, |B|) & \{|C|=|D|\} \\
             & = |C| - max(|A|, |B|) & \{\text{根据式(\ref{eq:rl-dx-dy}): $\delta'(x)=-1 \Rightarrow |B|-|A|=-1$} \} \\
             & = |C| - (|B| + 1) & \{ \delta(y) = 1 \Rightarrow |C|-|B|=1\} \\
             & = 0
  \end{array}
\]

\item 若$\delta(y)=-1$,根据式(\ref{eq:rl-dx-dy}),我们有$\delta'(x)=0 \Rightarrow |A|=|B|$。

\[
  \begin{array}{rll}
  \delta'(y) & = max(|C|, |D|) - max(|A|, |B|) & \{|A|=|B|\} \\
             & = max(|C|, |D|) - |B| & \{ \text{根据式(\ref{eq:rl-dz-dy}): $|D|-|C|=1$} \} \\
             & = |C| + 1 - |B| & \{  \delta(y) = -1 \Rightarrow |C|-|B|=-1\} \\
             & = 0
  \end{array}
\]

\end{itemize}

全部三种情况的结果都是$\delta'(y)=0$。

将上述结果归纳起来,可以得到新的平衡因子如下:

\be
  \begin{array}{l}
  \delta'(x) = \left \{
    \begin{array}
    {r@{\quad:\quad}l}
    -1 & \delta(y) = 1 \\
    0 & otherwise
    \end{array}
    \right. \\
  \delta'(y) = 0 \\
  \delta'(z) = \left \{
    \begin{array}
    {r@{\quad:\quad}l}
    1 & \delta(y) = -1 \\
    0 & otherwise
    \end{array}
    \right.
  \end{array}
  \label{eq:rl-result}
\ee

\subsubsection{左-右偏(Left-right lean)的情况}

左-右偏的情况和右-左偏的情况对称。使用类似的推导,我们可以得到和式(\ref{eq:rl-result})完全相同的结果。

\subsection{模式匹配}
各种修复平衡的情况可以抽象成模式,下面的函数使用模式匹配定义了平衡修复算法。

\be
balance(T, \Delta H) = \left \{
  \begin{array}
  {r@{\quad:\quad}l}
  (((A, x, B, \delta(x)), y, (C, z, D, 0), 0), 0) & P_{ll}(T) \\
  (((A, x, B, 0), y, (C, z, D, \delta(z)), 0), 0) & P_{rr}(T) \\
  (((A, x, B, \delta'(x)), y, (C, z, D, \delta'(z)), 0), 0) & P_{rl}(T) \lor P_{lr}(T) \\
  (T, \Delta H) & otherwise
  \end{array}
\right.
\ee

其中$P_{ll}(T)$表示树$T$满足左-左偏的情况。$\delta'(x)$和$delta'(z)$按照式(\ref{eq:rl-result})定义。

\be
\begin{array}{l}
P_{ll}(T): T = (((A, x, B, \delta(x)), y, C, -1), z, D, -2) \\
P_{rr}(T): T = (A, x, (B, y, (C, z, D, \delta(z)), 1), 2) \\
P_{rl}(T): T = ((A, x, (B, y, C, \delta(y)), 1), z, D, -2) \\
P_{lr}(T): T = (A, x, ((B, y, C, \delta(y)), z, D, -1), 2)
\end{array}
\ee

下面的Haskell例子代码实现了这一平衡修复函数。

\begin{lstlisting}
balance (Br (Br (Br a x b dx) y c (-1)) z d (-2), _) =
        (Br (Br a x b dx) y (Br c z d 0) 0, 0)
balance (Br a x (Br b y (Br c z d dz)    1)    2, _) =
        (Br (Br a x b 0) y (Br c z d dz) 0, 0)
balance (Br (Br a x (Br b y c dy)    1) z d (-2), _) =
        (Br (Br a x b dx') y (Br c z d dz') 0, 0) where
    dx' = if dy ==  1 then -1 else 0
    dz' = if dy == -1 then  1 else 0
balance (Br a x (Br (Br b y c dy) z d (-1))    2, _) =
        (Br (Br a x b dx') y (Br c z d dz') 0, 0) where
    dx' = if dy ==  1 then -1 else 0
    dz' = if dy == -1 then  1 else 0
balance (t, d) = (t, d)
\end{lstlisting}

插入算法的性能和树的高度成正比,根据之前给出的证明,如果AVL树包含$n$个元素,插入算法的性能为$O(\lg n)$。

\subsubsection{验证}
\index{AVL tree!verification}

可以定义一个函数来检查一棵树是否是AVL树。我们需要验证两方面:首先它必须是一棵合法的二叉搜索树;其次它满足AVL树的性质。

我们略过二叉搜索树的检验,把它留给读者作为练习。

为了验证AVL树的性质是否满足,我们需要检查左右分支的高度差,然后再递归检查左右分支是否也满足AVL树的性质。直到最终到达叶子节点。

\be
  avl?(T) = \left \{
  \begin{array}
  {r@{\quad:\quad}l}
  True & T = \phi \\
  avl?(L) \land avl?(R) \land ||R|-|L|| \leq 1 & otherwise
  \end{array}
  \right .
\ee

树的高度可以根据定义递归进行计算:

\be
  |T| = \left \{
  \begin{array}
  {r@{\quad:\quad}l}
  0 & T = \phi \\
  1 + max(|R|, |L|) & otherwise
  \end{array}
  \right .
\ee

相应的Haskell例子程序实现如下:

\begin{lstlisting}
isAVL Empty = True
isAVL (Br l _ r d) = and [isAVL l, isAVL r, abs (height r - height l) <= 1]

height Empty = 0
height (Br l _ r _) = 1 + max (height l) (height r)
\end{lstlisting}

\begin{Exercise}
编写程序检查一棵二叉树是否是二叉搜索树。如果使用命令式(imperative)语言,请考虑如何消除递归。
\end{Exercise}



% ================================================================
%                 Deletion
% ================================================================

\section{删除}
\index{AVL树!删除}

我们在二叉搜索树的章节曾经解释过,在纯函数式的环境中删除操作意义不大。由于树是只读的,它通常是在一次性构建之后用于反复查询。

我们曾经在红黑树一章中实现了删除,它本质上是重新构建一棵新树。我们将类似的AVL树删除实现留给读者作为练习。

\begin{Exercise}

\begin{itemize}
\item 参考红黑树的函数式删除算法,编程实现AVL树的删除操作。
\end{itemize}

\end{Exercise}

\section{AVL树的命令式算法$\star$}
\index{AVL树!命令式插入}

我们已经介绍了AVL树相关的主要内容。本节我们展示传统的AVL树“插入-旋转”算法,读者可以将其和模式匹配算法进行比较。

和红黑树的命令式插入算法相似,我们先按照普通二叉搜索树将新元素插入,然后再通过旋转操作恢复平衡。

\begin{algorithmic}[1]
\Function{Insert}{$T, k$}
  \State $root \gets T$
  \State $x \gets$ \Call{Create-Leaf}{$k$}
  \State \Call{$\delta$}{$x$} $\gets 0$
  \State $parent \gets$ NIL
  \While{$T \neq$ NIL}
    \State $parent \gets T$
    \If{$k <$ \Call{Key}{$T$}}
      \State $T \gets $ \Call{Left}{$T$}
    \Else
      \State $T \gets $ \Call{Right}{$T$}
    \EndIf
  \EndWhile
  \State \Call{Parent}{$x$} $\gets parent$
  \If{$parent =$ NIL} \Comment{树$T$为空}
    \State \Return $x$
  \ElsIf{$k <$ \Call{Key}{$parent$}}
    \State \Call{Left}{$parent$} $\gets x$
  \Else
    \State \Call{Right}{$parent$} $\gets x$
  \EndIf
  \State \Return \Call{AVL-Insert-Fix}{$root, x$}
\EndFunction
\end{algorithmic}

插入新元素后,树的高度可能增加,因此平衡因子$\delta$也会变化。插入到右侧会使$\delta$增加1,插入左侧会使$\delta$减少1。在算法结束前,我们需要从$x$开始,自底向上修复平衡,直到根节点。

下面的Python例子程序实现了插入算法的主要部份\footnote{相应的C和C++例子代码可以和本书一起获得。}。
\lstset{language=Python}
\begin{lstlisting}
def avl_insert(t, key):
    root = t
    x = Node(key)
    parent = None
    while(t):
        parent = t
        if(key < t.key):
            t = t.left
        else:
            t = t.right
    if parent is None: #tree is empty
        root = x
    elif key < parent.key:
        parent.set_left(x)
    else:
        parent.set_right(x)
    return avl_insert_fix(root, x)
\end{lstlisting}

算法首先自顶向下从根开始搜索插入位置,然后将新元素作为叶子节点插入。最后它调用修复程序,并传入根和新插入的节点。

这里我们复用了红黑树一章中定义的\texttt{set\_left()}和\texttt{set\_right()}方法。

为了修复平衡,我们需要检查新节点是插入到了左侧还是右侧。如果在左侧,平衡因子$\delta$减小,否则增加。记新的平衡因子为$\delta'$,我们有如下三种情况:

\begin{itemize}
\item 若$|\delta| = 1$而$|\delta'| = 0$,说明插入后树处于平衡状态。父节点的高度没有发生变化,算法结束。

\item 若$|\delta| = 0$而$|\delta'| = 1$,说明左右分支之一的高度增加了,我们需要继续向上检查树的平衡性。

\item 若$|\delta| = 1$ and $|\delta'| = 2$,说明AVL树不再平衡了,我们需要进行旋转操作进行修复。
\end{itemize}

\begin{algorithmic}[1]
\Function{AVL-Insert-Fix}{$T, x$}
  \While{\Call{Parent}{$x$} $\neq$ NIL}
    \State $\delta \gets $ \textproc{$\delta$}(\Call{Parent}{$x$})
    \If{$x = $ \textproc{Left}(\Call{Parent}{$x$})}
      \State $\delta' \gets \delta - 1$
    \Else
      \State $\delta' \gets \delta + 1$
    \EndIf
    \State \textproc{$\delta$}(\Call{Parent}{$x$}) $\gets \delta'$
    \State $P \gets $ \Call{Parent}{$x$}
    \State $L \gets $ \Call{Left}{$x$}
    \State $R \gets $ \Call{Right}{$x$}
    \If{$|\delta| = 1$ and $|\delta'| = 0$} \Comment{高度没有变化,结束。}
      \State \Return $T$
    \ElsIf{$|\delta| = 0$ and $|\delta'| = 1$} \Comment{继续自底向上进行更新。}
      \State $x \gets P$
    \ElsIf{$|\delta| = 1$ and $|\delta'| = 2$}
      \If{$\delta'=2$}
        \If{$\delta(R) = 1$} \Comment{右-右情况}
          \State $\delta(P) \gets 0$ \Comment{根据式(\ref{eq:rr-result})}
          \State $\delta(R) \gets 0$
          \State $T \gets $ \Call{Left-Rotate}{$T, P$}
        \EndIf
        \If{$\delta(R) = -1$} \Comment{右-左情况}
          \State $\delta_y \gets $ \textproc{$\delta$}(\Call{Left}{$R$}) \Comment{根据式(\ref{eq:rl-result})}
          \If{$\delta_y = 1$}
            \State $\delta(P) \gets -1$
          \Else
            \State $\delta(P) \gets 0$
          \EndIf
          \State \textproc{$\delta$}(\Call{Left}{$R$}) $\gets 0$
          \If{$\delta_y = -1$}
            \State $\delta(R) \gets 1$
          \Else
            \State $\delta(R) \gets 0$
          \EndIf
          \State $T \gets $ \Call{Right-Rotate}{$T, R$}
          \State $T \gets $ \Call{Left-Rotate}{$T, P$}
        \EndIf
      \EndIf
      \If{$\delta' = -2$}
        \If{$\delta(L) = -1$} \Comment{左-左情况}
          \State $\delta(P) \gets 0$
          \State $\delta(L) \gets 0$
          \State \Call{Right-Rotate}{$T, P$}
        \Else \Comment{左-右情况}
          \State $\delta_y \gets $ \textproc{$\delta$}(\Call{Right}{$L$})
          \If{$\delta_y = 1$}
            \State $\delta(L) \gets -1$
          \Else
            \State $\delta(L) \gets 0$
          \EndIf
          \State \textproc{$\delta$}(\Call{Right}{$L$}) $\gets 0$
          \If{$\delta_y = -1$}
            \State $\delta(P) \gets 1$
          \Else
            \State $\delta(P) \gets 0$
          \EndIf
          \State \Call{Left-Rotate}{$T, L$}
          \State \Call{Right-Rotate}{$T, P$}
        \EndIf
      \EndIf
      \State break
    \EndIf
  \EndWhile
  \State \Return $T$
\EndFunction
\end{algorithmic}

这里我们复用了红黑树一章中定义的旋转操作。单纯旋转操作并不更新平衡因子$\delta$。由于旋转变换改变了树结构,增加了平衡性,因此我们需要重新计算平衡因子。这里直接使用了上面的结果。在4种情况中,右-右偏和左-左偏需要进行一次旋转;而右-左偏和左-右偏需要进行两次旋转。

相关的Python例子程序如下:

\begin{lstlisting}
def avl_insert_fix(t, x):
    while x.parent is not None:
        d2 = d1 = x.parent.delta
        if x == x.parent.left:
            d2 = d2 - 1
        else:
            d2 = d2 + 1
        x.parent.delta = d2
        (p, l, r) = (x.parent, x.parent.left, x.parent.right)
        if abs(d1) == 1 and abs(d2) == 0:
            return t
        elif abs(d1) == 0 and abs(d2) == 1:
            x = x.parent
        elif abs(d1)==1 and abs(d2) == 2:
            if d2 == 2:
                if r.delta == 1:  # 右-右情况
                    p.delta = 0
                    r.delta = 0
                    t = left_rotate(t, p)
                if r.delta == -1: # 右-左情况
                    dy = r.left.delta
                    if dy == 1:
                        p.delta = -1
                    else:
                        p.delta = 0
                    r.left.delta = 0
                    if dy == -1:
                        r.delta = 1
                    else:
                        r.delta = 0
                    t = right_rotate(t, r)
                    t = left_rotate(t, p)
            if d2 == -2:
                if l.delta == -1: # 左-左情况
                    p.delta = 0
                    l.delta = 0
                    t = right_rotate(t, p)
                if l.delta == 1: # 左-右情况
                    dy = l.right.delta
                    if dy == 1:
                        l.delta = -1
                    else:
                        l.delta = 0
                    l.right.delta = 0
                    if dy == -1:
                        p.delta = 1
                    else:
                        p.delta = 0
                    t = left_rotate(t, l)
                    t = right_rotate(t, p)
            break
    return t
\end{lstlisting}

我们略过AVL树的删除算法,留给读者作为练习。

\begin{Exercise}

\begin{itemize}
\item 编程实现AVL树的命令式(imperative)删除算法。
\end{itemize}

\end{Exercise}


\section{小结}
AVL树是在1962年由Adelson-Velskii和Landis\cite{wiki-avl}、\cite{TFATP}发表的。AVL树的命名来自两位作者的名字。它的历史要比红黑树更早。

人们很自然会比较AVL树和红黑树。它们都是自平衡二叉搜索树,对于主要的树操作,它们的性能都是$O(\lg n)$的。根据式(\ref{eq:AVL-height})的结果,AVL树的平衡性更为严格,因此在频繁查询的情况下,其表现要好于红黑树\cite{wiki-avl}。但红黑树在频繁插入和删除的情况下性能更佳。

很多流行的程序库使用红黑树作为自平衡二叉搜索树的内部实现,例如STL,AVL树同样也可以直观、高效地解决平衡问题。

在接下来的章节里,我们将介绍一些数据结构,它们使用边,而不是节点来存储信息,如Trie和Patricia等等。在保证平衡的情况下,如果允许含有两个以上的子分支,我们就得到了另一种有趣的数据结构-B树。

\ifx\wholebook\relax \else
\begin{thebibliography}{99}

\bibitem{hackage-avl}
Data.Tree.AVL http://hackage.haskell.org/packages/archive/AvlTree/4.2/doc/html/Data-Tree-AVL.html

\bibitem{okasaki}
Chris Okasaki. ``FUNCTIONAL PEARLS Red-Black Trees in a Functional Setting''. J. Functional Programming. 1998

\bibitem{wiki-avl}
Wikipedia. ``AVL tree''. http://en.wikipedia.org/wiki/AVL\_tree

\bibitem{TFATP}
Guy Cousinear, Michel Mauny. ``The Functional Approach to Programming''. Cambridge University Press; English Ed edition (October 29, 1998). ISBN-13: 978-0521576819

\bibitem{py-avl}
Pavel Grafov. ``Implementation of an AVL tree in Python''. http://github.com/pgrafov/python-avl-tree
\end{thebibliography}

\end{document}
\fi

% LocalWords:  AVL Okasaki STL
